\documentclass[a4paper,11pt]{article}

% Packages
\usepackage[utf8]{inputenc}
\usepackage[T1]{fontenc}
\usepackage[french]{babel}
\usepackage{fancyhdr}
\usepackage{geometry}
\usepackage{amsmath}
\usepackage{amssymb}
\usepackage{algorithm}
\usepackage{algpseudocode}

% Page layout
\geometry{a4paper, margin=2.5cm}

% Fix header height issue
\setlength{\headheight}{14pt}

% Header and footer setup
\pagestyle{fancy}
\fancyhf{} % Clear all header and footer fields

% Header
\fancyhead[R]{ISIMA - Théorie Des Jeux} % Right header

% Footer - with left, center, and right elements
\fancyfoot[L]{BOYA, HAFSOUNI} % Left footer
\fancyfoot[C]{Université de Clermont Auvergne} % Center footer
\fancyfoot[R]{\thepage} % Right footer

% Optional header/footer lines
\renewcommand{\headrulewidth}{0.4pt} % Header line (set to 0pt to remove)
\renewcommand{\footrulewidth}{0.4pt} % Footer line (set to 0pt to remove)

\begin{document}

\begin{center}
\Huge{ISIMA - Théorie Des Jeux}\\[0.5cm]
\LARGE{TP1 de Théorie Des Jeux}\\[0.2cm]
\Large{Decembre 2025}\\[0.1cm]
\Large{Erkin Tunc BOYA , Raed HAFSOUNI}
\end{center}

\section*{Exercice 1 -- Jeu des jetons colorés}

\subsection*{(a) Jeu à somme nulle}

À chaque coup, l’un des deux joueurs gagne et l’autre perd le même montant
(50, 40, 30 ou 0). Le gain total d’une partie est la somme des gains sur les
trois coups, et pour chaque issue la somme des gains de $A$ et $B$ est nulle.
Le jeu est donc à somme nulle.

\subsection*{(b) Matrice de gains}

Chaque joueur possède les trois jetons \textit{rouge, blanc, bleu} et doit
choisir un ordre d’utilisation de ces jetons sur trois coups. On numérote les
stratégies pures (permutations) comme suit :
\[
\begin{aligned}
1 &= (\text{rouge}, \text{blanc}, \text{bleu}),\\
2 &= (\text{rouge}, \text{bleu}, \text{blanc}),\\
3 &= (\text{blanc}, \text{rouge}, \text{bleu}),\\
4 &= (\text{blanc}, \text{bleu}, \text{rouge}),\\
5 &= (\text{bleu}, \text{rouge}, \text{blanc}),\\
6 &= (\text{bleu}, \text{blanc}, \text{rouge}).
\end{aligned}
\]

En appliquant les règles
(rouge bat blanc : $+50$, blanc bat bleu : $+40$, bleu bat rouge : $+30$,
même couleur : $0$),
on obtient la matrice de gain (pour $A$) :

\[
A =
\begin{array}{c|cccccc}
      & 1 & 2 & 3 & 4 & 5 & 6 \\ \hline
1 &   0  &   0  &   0  & 120 & -120 &   0  \\
2 &   0  &   0  & 120  &  0  &   0  & -120 \\
3 &   0  & -120 &   0  &  0  &   0  & 120  \\
4 & -120 &   0  &   0  &  0  & 120  &  0   \\
5 & 120  &   0  &   0  & -120&   0  &  0   \\
6 &  0   & 120  & -120 &  0  &   0  &  0
\end{array}
\]

Pour chaque ligne $i$, le joueur $B$ peut forcer un résultat $-120$ pour $A$,
donc
\[
\min_j a_{ij} = -120 \quad \Rightarrow \quad V^- = \max_i \min_j a_{ij} = -120.
\]

Pour chaque colonne $j$, le joueur $A$ peut forcer un résultat $+120$,
donc
\[
\max_i a_{ij} = 120 \quad \Rightarrow \quad V^+ = \min_j \max_i a_{ij} = 120.
\]

Comme
\[
V^- \neq V^+,
\]
il n’y a pas de point selle et pas de solution en stratégies pures.

\subsection*{(c) Stratégies mixtes et résolution par PL}

On introduit une stratégie mixte pour $A$ :
\[
X = (x_1,\ldots,x_6), \qquad x_i \ge 0,\quad \sum_{i=1}^{6} x_i = 1,
\]
et une valeur garantie $g$. Le programme linéaire du joueur $A$ est :
\[
\max g
\]
sous les contraintes
\[
E(X,j) = \sum_{i=1}^{6} a_{ij} x_i \;\ge\; g, \quad j=1,\ldots,6,
\]
et
\[
\sum_{i=1}^{6} x_i = 1,\quad x_i \ge 0.
\]

La résolution numérique avec OPL donne
\[
g^{*} = 0,
\qquad
x_1^{*} = x_4^{*} = x_5^{*} = \frac{1}{3},
\qquad
x_2^{*} = x_3^{*} = x_6^{*} = 0.
\]

Donc la stratégie optimale de $A$ est
\[
X^{*} =
\left(
\frac{1}{3},\;
0,\;
0,\;
\frac{1}{3},\;
\frac{1}{3},\;
0
\right),
\qquad
v = 0.
\]

Par symétrie et par résolution analogue du programme du joueur $B$, on obtient
une stratégie optimale de même forme pour $B$, et la valeur du jeu est
également $0$ pour lui (jeu équilibré).

\subsection*{(d) Lien avec les trois combinaisons proposées}

Dans l’énoncé, on propose au joueur maximisateur de jouer aussi souvent les
trois combinaisons suivantes :
\begin{itemize}
    \item Bleu – Rouge – Blanc,
    \item Blanc – Bleu – Rouge,
    \item Rouge – Blanc – Bleu.
\end{itemize}

Dans notre numérotation, cela correspond exactement aux stratégies
\[
5 = (\text{bleu}, \text{rouge}, \text{blanc}), \quad
4 = (\text{blanc}, \text{bleu}, \text{rouge}), \quad
1 = (\text{rouge}, \text{blanc}, \text{bleu}).
\]

Jouer chacune de ces trois stratégies avec probabilité $\frac{1}{3}$ revient donc
à utiliser la stratégie mixte
\[
X = \left(
\frac{1}{3},\;
0,\;
0,\;
\frac{1}{3},\;
\frac{1}{3},\;
0
\right),
\]
qui est exactement la solution optimale $X^{*}$ trouvée par le programme linéaire.

Pour cette stratégie, l’espérance de gain de $A$ contre chaque stratégie pure
de $B$ est nulle :
\[
E(X,j) = 0,\quad \forall j=1,\ldots,6,
\]
et le joueur $A$ garantit donc la valeur du jeu
\[
v = 0.
\]


%------------------------------------------------------


\section*{Exercice 2 -- Jeu de Domino}

\subsection*{(a) Jeu à somme nulle}

On considère une grille $2\times 3$ (6 cases blanches) et les 7 configurations possibles
d’un domino couvrant deux cases. Le joueur $X$ choisit une configuration, le joueur $Y$
choisit simultanément une case parmi les 6.

\begin{itemize}
  \item Si la case choisie par $Y$ est recouverte par la configuration de $X$, alors $Y$ gagne.
  \item Sinon, $X$ gagne.
\end{itemize}

On code le gain de $X$ par
\[
\text{gain}(X) =
\begin{cases}
+1 & \text{si $X$ gagne},\\[1mm]
-1 & \text{si $Y$ gagne}.
\end{cases}
\]
Dans tous les cas, le gain de $Y$ vaut $-\text{gain}(X)$, donc la somme des gains est toujours nulle.
Le jeu est donc bien un jeu à somme nulle.

\subsection*{(b) Matrice de gains}

Les 7 configurations possibles de placement du domino (numérotées de 1 à 7) sont les stratégies
pures de $X$. Les 6 cases de la grille (numérotées de 1 à 6) sont les stratégies pures de $Y$.

On note $a_{ij}$ le gain de $X$ lorsque $X$ joue la configuration $i$ et $Y$ annonce la case $j$.
On obtient la matrice de gains suivante (gain pour $X$) :
\[
A =
\begin{pmatrix}
-1 &  1 &  1 & -1 &  1 &  1 \\
 1 &  1 & -1 &  1 &  1 & -1 \\
 1 & -1 &  1 &  1 & -1 &  1 \\
 1 & -1 & -1 &  1 &  1 &  1 \\
-1 & -1 &  1 &  1 &  1 &  1 \\
 1 &  1 &  1 &  1 & -1 & -1 \\
 1 &  1 &  1 & -1 & -1 &  1
\end{pmatrix}.
\]

Pour chaque stratégie pure $i$ de $X$, $Y$ peut toujours choisir une colonne qui donne $-1$ à $X$,
donc
\[
\min_j a_{ij} = -1 \quad \Rightarrow \quad
V^- = \max_i \min_j a_{ij} = -1.
\]

Pour chaque stratégie pure $j$ de $Y$, $X$ peut choisir une ligne avec un gain $+1$, donc
\[
\max_i a_{ij} = 1 \quad \Rightarrow \quad
V^+ = \min_j \max_i a_{ij} = 1.
\]

Comme
\[
V^- \neq V^+,
\]
il n’y a pas de point selle en stratégies pures, donc pas de solution en stratégies pures.

\subsection*{(c) Stratégies mixtes et résolution par PL}

On introduit une stratégie mixte pour $X$ :
\[
X = (x_1,\ldots,x_7), \qquad
x_i \ge 0,\quad \sum_{i=1}^{7} x_i = 1.
\]

L’espérance de gain de $X$ contre la stratégie pure $j$ de $Y$ est
\[
E(X,j) = \sum_{i=1}^{7} a_{ij} x_i.
\]
Pour que $X$ se garantisse au moins un gain $g$, il faut
\[
E(X,j) \ge g, \qquad j = 1,\ldots,6.
\]

Le programme linéaire du joueur $X$ est donc :
\[
\max g
\]
sous les contraintes
\[
\sum_{i=1}^{7} a_{ij} x_i \;\ge\; g, \qquad j=1,\ldots,6,
\]
\[
\sum_{i=1}^{7} x_i = 1, \qquad x_i \ge 0.
\]

Ce modèle est exactement celui codé dans le fichier \texttt{Ex2.txt} (une contrainte par colonne
de la matrice $A$).

\medskip

La résolution numérique avec OPL donne :
\[
g^{*} = \frac{1}{3} \approx 0.3333,
\]
et la solution optimale
\[
x_1^{*} = x_4^{*} = x_6^{*} = \frac{1}{3}, \qquad
x_2^{*} = x_3^{*} = x_5^{*} = x_7^{*} = 0.
\]

Ainsi, une stratégie optimale pour le joueur $X$ est
\[
X^{*} =
\left(
\frac{1}{3},\;
0,\;
0,\;
\frac{1}{3},\;
0,\;
\frac{1}{3},\;
0
\right).
\]

Dans cette stratégie, seules les configurations $1$, $4$ et $6$ sont utilisées, chacune avec
probabilité $1/3$. On vérifie que, pour chaque case $j$ choisie par $Y$,
\[
E(X^{*},j) = \frac{1}{3},
\]
donc $X$ se garantit un gain espéré de $1/3$ quelle que soit la stratégie pure de $Y$.

Par symétrie et via le programme linéaire dual pour $Y$, on obtient une stratégie mixte
optimale $Y^{*}$ telle que la valeur du jeu soit également $1/3$ pour $X$.

\subsection*{Conclusion : valeur du jeu et joueur avantagé}

La valeur du jeu (gain pour $X$) est
\[
v = g^{*} = \frac{1}{3} > 0.
\]

Le joueur $X$ (qui choisit la configuration de dominos) est donc avantagé : il peut garantir un
gain espéré strictement positif, tandis que $Y$ ne peut faire mieux que limiter ce gain à $1/3$.




\section*{Exercice 3 -- Deux produits concurrents}

\subsection*{Explication du code de l'exercice 3}

Dans ce code,on commence par définir les stratégies possibles pour les deux joueurs et on construit la matrice des gains $M$ selon les quantités $Q1$ et $Q2$.  
Les variables $x[i]$ et $y[i]$ servent à représenter les probabilités que chaque joueur choisisse ses stratégies, et $v$ correspond au gain minimum que le joueur A peut s’assurer.  
On cherche à maximiser $v$ en mettant des contraintes pour que ce gain soit garanti peu importe ce que fait le joueur B, et pour que les probabilités de chaque joueur somment à 1.  
À la fin, le programme affiche la valeur du jeu ainsi que les probabilités des stratégies réellement utilisées par les joueurs.

\subsection*{(a) Les 3 stratégies possibles pour chaque entreprise}

Chaque entreprise (A et B) doit choisir comment planifier l’amélioration de ses deux produits.

On considère les stratégies pures suivantes :

\begin{enumerate}
    \item \textbf{S (Simultané)} : améliorer les deux produits en même temps. Les deux produits sont disponibles au bout de 12 mois.
    \item \textbf{12 (Produit 1 puis 2)} : l’entreprise améliore d’abord le produit 1, puis le produit 2.
    \item \textbf{21 (Produit 2 puis 1)} : l’entreprise améliore d’abord le produit 2, puis le produit 1.
\end{enumerate}

Les écarts de dates de lancement entre A et B, pour chaque produit, donnent les gains/pertes de A en pourcentage du volume futur de ce produit, selon la règle de l’énoncé :
\begin{itemize}
    \item si A et B lancent simultanément un produit : gain de $8\%$ pour A,
    \item si A est en avance de $2, 6$ ou $8$ mois : gain de $20\%, 30\%$ ou $40\%$,
    \item si B est en avance de $1, 3, 7$ ou $10$ mois : perte de $4\%, 10\%, 12\%, 14\%$ pour A.
\end{itemize}

\subsection*{(b) Matrices de gains par produit}

On note $G^{(1)}$ la matrice des gains (en fraction de $Q_1$) pour le produit~1, et $G^{(2)}$ la matrice des gains (en fraction de $Q_2$) pour le produit~2.  
En utilisant les écarts de temps entre lancements selon les stratégies ci-dessus et les pourcentages donnés, on obtient les matrices suivantes pour A (lignes = A, colonnes = B, stratégies dans l’ordre $S,12,21$) :

\medskip
\noindent\textbf{Produit 1 -- gains de A (proportion de $Q_1$)}
\[
G^{(1)} =
\begin{array}{c|ccc}
      & S   & 12   & 21   \\ \hline
S   & 0{,}08  & 0{,}20  & 0{,}30  \\
12  & 0{,}20  & -0{,}04 & 0{,}40  \\
21  & -0{,}12 & -0{,}14 & -0{,}04
\end{array}
\]

\medskip
\noindent\textbf{Produit 2 -- gains de A (proportion de $Q_2$)}
\[
G^{(2)} =
\begin{array}{c|ccc}
      & S   & 12    & 21    \\ \hline
S   & 0{,}08  & 0{,}30  & -0{,}10 \\
12  & -0{,}12 & -0{,}04 & -0{,}14 \\
21  & 0{,}20  & 0{,}40  & -0{,}04
\end{array}
\]

Ces coefficients utilisent exactement les pourcentages $\{8, 20, 30, 40, -4, -10, -12, -14\}$ selon que A est en avance ou en retard sur B pour le produit considéré.

\subsection*{(c) Jeu à somme nulle et matrice globale des gains}

Comme il n’y a que deux entreprises et que le marché est intégralement partagé, toute augmentation de part de marché de A est une perte symétrique pour B : le jeu est donc \emph{à somme nulle}.

On note $Q_1$ et $Q_2$ les quantités totales vendues respectivement du produit~1 et du produit~2.  
Le gain total (en volume) pour A, lorsque A joue la stratégie $i$ et B la stratégie $j$, est
\[
G(i,j) = G^{(1)}(i,j)\,Q_1 + G^{(2)}(i,j)\,Q_2.
\]

En combinant les deux matrices précédentes, on obtient la matrice globale suivante (lignes = A, colonnes = B, dans l’ordre $S,12,21$) :

\[
G =
\begin{array}{c|ccc}
      & S & 12 & 21 \\ \hline
S   & 0{,}08 Q_1 + 0{,}08 Q_2 & 0{,}20 Q_1 + 0{,}30 Q_2 & 0{,}30 Q_1 - 0{,}10 Q_2 \\
12  & 0{,}20 Q_1 - 0{,}12 Q_2 & -0{,}04(Q_1 + Q_2)       & 0{,}40 Q_1 - 0{,}14 Q_2 \\
21  & -0{,}12 Q_1 + 0{,}20 Q_2 & -0{,}14 Q_1 + 0{,}40 Q_2 & -0{,}04(Q_1 + Q_2)
\end{array}
\]

C’est la matrice des gains du jeu à somme nulle (A maximisateur, B minimisateur).

\subsection*{(d) Existence d’un point selle}

On cherche un point selle en stratégies pures pour deux cas particuliers.

\subsubsection*{Cas i) \boldmath$Q_1 = Q_2 = x$}

On prend $Q_1 = Q_2 = x$. La matrice devient
\[
G_x = x \cdot
\begin{pmatrix}
0{,}16 & 0{,}50 & 0{,}20 \\
0{,}08 & -0{,}08 & 0{,}26 \\
0{,}08 & 0{,}26 & -0{,}08
\end{pmatrix}.
\]

Pour A (maximiseur), on regarde les minima par ligne :
\[
\min \text{ligne}_1 = 0{,}16x,\quad
\min \text{ligne}_2 = -0{,}08x,\quad
\min \text{ligne}_3 = -0{,}08x,
\]
d’où
\[
V^- = \max_i \min_j G_x(i,j) = 0{,}16x.
\]

Pour B (minimiseur), on regarde les maxima par colonne :
\[
\max \text{col}_1 = 0{,}16x,\quad
\max \text{col}_2 = 0{,}50x,\quad
\max \text{col}_3 = 0{,}26x,
\]
d’où
\[
V^+ = \min_j \max_i G_x(i,j) = 0{,}16x.
\]

Ainsi
\[
V^- = V^+ = 0{,}16x,
\]
et il existe un \textbf{point selle} en stratégies pures.  
Il est atteint pour la stratégie \textbf{S} des deux entreprises (ligne 1, colonne 1).  

En résolvant le programme linéaire associé (modèle OPL avec $Q_1 = Q_2 = 1$), on obtient bien
\[
g^{*} = 0{,}16, \qquad X^{*} = (1,0,0),
\]
ce qui confirme que la stratégie simultanée $S$ est optimale pour A et que la valeur du jeu vaut
\[
v = 0{,}16x.
\]

\subsubsection*{Cas ii) \boldmath$Q_1 = Q_2/2 = x$}

Ici $Q_1 = x$ et $Q_2 = 2x$. En remplaçant dans la matrice globale, on obtient
\[
G_x' = x \cdot
\begin{pmatrix}
0{,}24 & 0{,}80 & 0{,}10 \\
-0{,}04 & -0{,}12 & 0{,}12 \\
0{,}28 & 0{,}66 & -0{,}12
\end{pmatrix}.
\]

Pour A (maximiseur), les minima par ligne sont
\[
\min \text{ligne}_1 = 0{,}10x,\quad
\min \text{ligne}_2 = -0{,}12x,\quad
\min \text{ligne}_3 = -0{,}12x,
\]
donc
\[
V^- = \max_i \min_j G_x'(i,j) = 0{,}10x.
\]

Pour B (minimiseur), les maxima par colonne sont
\[
\max \text{col}_1 = 0{,}28x,\quad
\max \text{col}_2 = 0{,}80x,\quad
\max \text{col}_3 = 0{,}12x,
\]
donc
\[
V^+ = \min_j \max_i G_x'(i,j) = 0{,}12x.
\]

On a donc
\[
V^- = 0{,}10x \quad\text{et}\quad V^+ = 0{,}12x \quad\Rightarrow\quad V^- \neq V^+.
\]

En résolvant numériquement le programme linéaire (modèle OPL avec $Q_1 = 1$ et $Q_2 = 2$), on trouve
\[
g^{*} \approx 0{,}10933,
\qquad
X^{*} \approx (0{,}53333,\; 0{,}46667,\; 0).
\]

La valeur du jeu vérifie bien
\[
V^- < g^{*} < V^+,
\]
ce qui confirme qu'il n'existe pas de point selle en stratégies pures dans ce cas.  
Aucune paire de stratégies pures $(\text{stratégie de A}, \text{stratégie de B})$ n’est stable : chaque joueur peut améliorer son gain en déviant si l’autre maintient sa stratégie, et A doit utiliser une stratégie mixte pour garantir un gain espéré d’environ $0{,}1093x$.

\end{document}
