\documentclass[a4paper,11pt]{article}

% Packages
\usepackage[utf8]{inputenc}
\usepackage[T1]{fontenc}
\usepackage[french]{babel}
\usepackage{fancyhdr}
\usepackage{geometry}
\usepackage{amsmath}
\usepackage{amssymb}
\usepackage{algorithm}
\usepackage{algpseudocode}

% Page layout
\geometry{a4paper, margin=2.5cm}

% Fix header height issue
\setlength{\headheight}{14pt}

% Header and footer setup
\pagestyle{fancy}
\fancyhf{} % Clear all header and footer fields

% Header
\fancyhead[R]{ISIMA - Théorie Des Jeux} % Right header

% Footer - with left, center, and right elements
\fancyfoot[L]{BOYA, HAFSOUNI} % Left footer
\fancyfoot[C]{Université de Clermont Auvergne} % Center footer
\fancyfoot[R]{\thepage} % Right footer

% Optional header/footer lines
\renewcommand{\headrulewidth}{0.4pt} % Header line (set to 0pt to remove)
\renewcommand{\footrulewidth}{0.4pt} % Footer line (set to 0pt to remove)

\begin{document}

\begin{center}
\Huge{ISIMA - Théorie Des Jeux}\\[0.5cm]
\LARGE{TP2 de Théorie Des Jeux}\\[0.2cm]
\Large{Decembre 2025}\\[0.1cm]
\Large{Erkin Tunc BOYA , Raed HAFSOUNI}
\end{center}

%-------------------------------------------------------------

\section*{Exercice 1 -- Équilibres de Nash en stratégies pures}

On considère deux joueurs et deux matrices de gains $A$ et $B$, où :
\begin{itemize}
  \item $A[i][j]$ représente le gain du joueur~I,
  \item $B[i][j]$ représente le gain du joueur~II,
\end{itemize}
lorsque le joueur~I joue la stratégie $i$ et le joueur~II joue la stratégie $j$.

\paragraph{Méthode de résolution (meilleure réponse).}
Un couple de stratégies $(i,j)$ est un équilibre de Nash en stratégies pures si :
\[
i \in \arg\max_{i'} A[i'][j]
\quad \text{et} \quad
j \in \arg\max_{j'} B[i][j'].
\]
Autrement dit, aucun joueur n’a intérêt à dévier unilatéralement de sa stratégie.

La méthode utilisée consiste à :
\begin{enumerate}
  \item déterminer, pour chaque stratégie de l’adversaire, l’ensemble des meilleures réponses,
  \item identifier les couples de stratégies qui sont simultanément des meilleures réponses pour les deux joueurs.
\end{enumerate}

\paragraph{Résultats de TD3 .}

\begin{itemize}
  \item \textbf{Exercice 1 :}  
  l’algorithme met en évidence deux équilibres de Nash en stratégies pures :
  \[
  (0,0) \quad \text{et} \quad (1,1).
  \]

  \item \textbf{Exercice 2 :}  
  l’algorithme met en évidence deux équilibres de Nash en stratégies pures :
  \[
  (0,1) \quad \text{et} \quad (1,0).
  \]
\end{itemize}

\paragraph{Remarque.}
Les indices sont donnés en base 0, conformément à la convention informatique utilisée.
Pour une écriture mathématique standard, il suffit d’ajouter $1$ à chaque indice.



%---------------------------------------------------------

\section*{Exercice 2}

\subsection{Jeu Gauche -- Droite -- Haut -- Bas}

On considère un jeu à deux joueurs étudié sous forme extensive et sous forme normale.

\paragraph{Forme extensive.}
Le joueur~1 joue en premier et choisit entre \textbf{Haut (H)} et \textbf{Bas (B)}.
Le joueur~2 observe ce choix et décide ensuite entre \textbf{Gauche (G)} et \textbf{Droite (D)}.
Les gains associés aux issues sont donnés par l’arbre du jeu.

\paragraph{Forme normale.}
Une stratégie du joueur~2 est un plan complet indiquant son action après chaque choix possible du joueur~1.
Ainsi :
\[
S_1 = \{H,B\}, \qquad
S_2 = \{GG, GD, DG, DD\}.
\]

La matrice des gains correspondante est :
\[
\begin{array}{c|cccc}
      & GG & GD & DG & DD \\
\hline
H & (2,1) & (2,1) & (3,-2) & (3,-2) \\
B & (-6,2) & (-1,4) & (-6,2) & (-1,4)
\end{array}
\]

\paragraph{Équilibres.}
À l’aide de Gambit, on obtient plusieurs équilibres de Nash en forme normale.
En revanche, l’unique équilibre parfait en sous-jeux est :
\[
(H,\,GD),
\]
ce qui conduit à l’issue \((H,G)\) avec gains \((2,1)\).


\subsection{Jeu de production (entrée sur le marché)}

\paragraph{Forme normale.}
La forme normale du jeu est donnée par la matrice suivante, où le premier gain est celui de la firme~A et le second celui de la firme~B :
\[
\begin{array}{c|cc}
      & \text{Entrer} & \text{Ne pas entrer} \\
\hline
\text{Accepter} & (3,2) & (4,0) \\
\text{Refuser}  & (-2,-3) & (4,0)
\end{array}
\]


\paragraph{Équilibres en forme normale.}
À l’aide de Gambit, on obtient les équilibres de Nash suivants :
\begin{itemize}
  \item un équilibre pur : \((\text{Accepter},\,\text{Entrer})\) ;
  \item un équilibre pur : \((\text{Refuser},\,\text{Ne pas entrer})\) ;
  \item un équilibre mixte dans lequel la firme~A joue
  \(\text{Accepter}\) avec la probabilité \(3/5\) et la firme~B choisit
  \(\text{Ne pas entrer}\).
\end{itemize}

\paragraph{Forme extensive et équilibre parfait en sous-jeux.}
Le jeu est naturellement séquentiel : la firme~B décide d’abord d’entrer ou non sur le marché.
En cas d’entrée, la firme~A choisit entre accepter ou refuser.

Par raisonnement à rebours, si la firme~B entre, la firme~A préfère accepter l’entrée
(car \(3 > -2\)).
Anticipant ce comportement, la firme~B choisit d’entrer.

L’unique équilibre parfait en sous-jeux est donc :
\[
(\text{Entrer},\,\text{Accepter}),
\]
avec des gains \((3,2)\).

L’équilibre \((\text{Refuser},\,\text{Ne pas entrer})\) est un équilibre de Nash en forme normale,
mais il n’est pas parfait en sous-jeux car la menace de refus n’est pas crédible.

%-----------------------------------------------------------
\section*{Exercice 3}

\paragraph{1. Probabilité de gain du joueur I.}
Pour chaque choix initial du joueur~I, le joueur~II choisit rationnellement la roue
restante qui minimise la probabilité de gain du joueur~I.

\begin{itemize}
  \item Si le joueur~I choisit la roue~1, sa probabilité de gain est \(\frac{4}{9}\).
  \item S’il choisit la roue~2, sa probabilité de gain est également \(\frac{4}{9}\).
  \item S’il choisit la roue~3, sa probabilité de gain est \(\frac{1}{3}\).
\end{itemize}

Ainsi, face à un adversaire rationnel, le joueur~I maximise sa probabilité de gain
en choisissant la roue~1 ou la roue~2.

\paragraph{2. Forme extensive.}
Le jeu est séquentiel et à information parfaite.
Le joueur~I choisit d’abord une roue parmi \(R1, R2, R3\).
Après observation de ce choix, le joueur~II choisit l’une des deux roues restantes.
Les gains aux feuilles de l’arbre correspondent aux espérances de gain calculées à la
question~1.

\paragraph{3. Forme normale.}
La forme normale est obtenue à partir de la forme extensive à l’aide de Gambit.
Le joueur~I dispose de trois stratégies pures :
\[
S_1 = \{R1, R2, R3\}.
\]
Une stratégie du joueur~II est un plan complet indiquant la roue choisie après chaque
choix possible du joueur~I, ce qui conduit à \(2^3 = 8\) stratégies pures.

\paragraph{4. Équilibres de Nash.}
À l’aide de Gambit, on obtient des équilibres de Nash en stratégies pures.
Dans ces équilibres, le joueur~II choisit, pour chaque roue du joueur~I, la roue
restante qui minimise son espérance de gain.

Le joueur~I joue alors indifféremment la roue~1 ou la roue~2, chacune conduisant à un
gain espéré de \(-\frac{1}{9}\).
La roue~3 n’apparaît dans aucun équilibre de Nash car elle conduit à un gain espéré
strictement plus faible.

Ces équilibres sont également des équilibres parfaits en sous-jeux.



%-----------------------------------------------------------


\end{document}