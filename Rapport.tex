\documentclass[a4paper,11pt]{article}

% Packages
\usepackage[utf8]{inputenc}
\usepackage[T1]{fontenc}
\usepackage[french]{babel}
\usepackage{fancyhdr}
\usepackage{geometry}
\usepackage{amsmath}
\usepackage{amssymb}
\usepackage{algorithm}
\usepackage{algpseudocode}

% Page layout
\geometry{a4paper, margin=2.5cm}

% Fix header height issue
\setlength{\headheight}{14pt}

% Header and footer setup
\pagestyle{fancy}
\fancyhf{} % Clear all header and footer fields

% Header
\fancyhead[R]{ISIMA - Théorie Des Jeux} % Right header

% Footer - with left, center, and right elements
\fancyfoot[L]{Erkin Tunc Boya} % Left footer
\fancyfoot[C]{Université de Clermont Auvergne} % Center footer
\fancyfoot[R]{\thepage} % Right footer

% Optional header/footer lines
\renewcommand{\headrulewidth}{0.4pt} % Header line (set to 0pt to remove)
\renewcommand{\footrulewidth}{0.4pt} % Footer line (set to 0pt to remove)

\begin{document}

\begin{center}
\Huge{ISIMA - Théorie Des Jeux}\\[0.5cm]
\LARGE{TP1 de Théorie Des Jeux}\\[0.2cm]
\Large{Decembre 2025}\\[0.1cm]
\Large{Erkin Tunc BOYA}
\end{center}

\section{Ex1}
Il y a deux joueurs $A$ et $B$. Chacun possède trois jetons : un rouge, un blanc et un bleu. 
C'est un jeu à deux joueurs, à somme nulle (si la valeur est $> 0$, c'est un gain pour $A$ et une perte pour $B$). 
Pour commencer, les joueurs sélectionnent chacun un de leurs jetons et les exhibent simultanément. 
Ils déterminent alors leur gain :

\begin{table}[h!]
    \centering
    \begin{tabular}{|c|c|}
        \hline
        \textbf{Lien entre couleurs} & \textbf{gain} \\ \hline
        rouge bat blanc              & 50  \\ 
        blanc bat bleu               & 40  \\ 
        bleu bat rouge               & 30  \\ 
        \text{même couleur}          & 0   \\ \hline
    \end{tabular}
\end{table}


\subsection{Modélisation du jeu}

On considère les stratégies pures suivantes pour chaque joueur (ordre dans lequel il utilise ses jetons) :

\subsubsection{Les stratégies}
\noindent
1 = (\text{rouge}, \text{blanc}, \text{bleu})\\
2 = (\text{rouge}, \text{bleu}, \text{blanc})\\
3 = (\text{blanc}, \text{rouge}, \text{bleu})\\
4 = (\text{blanc}, \text{bleu}, \text{rouge})\\
5 = (\text{bleu}, \text{rouge}, \text{blanc})\\
6 = (\text{bleu}, \text{blanc}, \text{rouge})\\[0.3cm]

La matrice de gains (pour le joueur $A$) est alors :
\[
\begin{array}{c|cccccc}
      & 1 & 2 & 3 & 4 & 5 & 6 \\ \hline
1 &  0   &  0   &   0   &  120  & -120 &   0   \\
2 &  0   &  0   &  120  &   0   &   0  & -120  \\
3 &  0   & -120 &   0   &   0   &   0  &  120  \\
4 & -120 &  0   &   0   &   0   &  120 &   0   \\
5 &  120 &  0   &   0   & -120  &   0  &   0   \\
6 &  0   & 120  & -120  &   0   &   0  &   0
\end{array}
\]


\subsection{Stratégie pure}

On calcule d'abord les bornes classiques :
\[
V^- = \max_i \min_j a_{ij}, 
\qquad 
V^+ = \min_j \max_i a_{ij}.
\]

Comme tous les minima des lignes valent $-120$, on obtient :
\[
V^- = \max(-120,-120,-120,-120,-120,-120) = -120.
\]

Comme tous les maxima des colonnes valent $120$, on obtient :
\[
V^+ = \min(120,120,120,120,120,120) = 120.
\]

Ainsi,
\[
V^- \neq V^+ \qquad \Longrightarrow \qquad 
\text{pas de point selle et pas de solution en stratégies pures.}
\]

\subsection{Stratégies mixtes}

Une stratégie mixte est un vecteur de probabilités.

Pour le joueur~A :
\[
X = (x_1, \ldots, x_6), \qquad 
x_i \ge 0 \ \text{pour tout } i = 1,\ldots,6, \qquad
\sum_{i=1}^{6} x_i = 1.
\]

Pour le joueur~B :
\[
Y = (y_1, \ldots, y_6), \qquad
y_j \ge 0 \ \text{pour tout } j = 1,\ldots,6, \qquad
\sum_{j=1}^{6} y_j = 1.
\]

\bigskip

Dans notre modélisation, il n'existe pas de point selle puisque
\[
V^- \neq V^+.
\]

Nous cherchons donc des stratégies mixtes \(X^*\) et \(Y^*\)
telles que l'espérance de gain du joueur~A soit maximisée et celle de B minimisée.

\medskip

Le système d'inégalités correspondant est :
\[ E(X,Y)
\begin{cases}
y_1(-120\,x_4 + 120\,x_5 ) \ge g, \\
y_2(-120\,x_3 + 120\,x_6 ) \ge g, \\
y_3( 120\,x_2 - 120\,x_6 ) \ge g, \\
y_4( 120\,x_1 - 120\,x_5 ) \ge g, \\
y_5(-120\,x_1 + 120\,x_4 ) \ge g, \\
y_6(-120\,x_2 + 120\,x_3 ) \ge g. \\
x_1 + x_2 + x_3 + x_4 + x_5 + x_6 = 1
\end{cases}
\]

% je vais mettre le reponce de code qu'on a ecrit




%------------------------------------------------------

\section{Ex2}

\subsection{Modelisation}
\[
A =
\begin{pmatrix}
-1 &  1 &  1 & -1 &  1 &  1 \\
 1 &  1 & -1 &  1 &  1 & -1 \\
 1 & -1 &  1 &  1 &  1 &  1 \\
 1 & -1 & -1 &  1 &  1 &  1 \\
-1 & -1 &  1 &  1 &  1 &  1 \\
 1 &  1 &  1 &  1 & -1 & -1 \\
 1 &  1 &  1 & -1 & -1 &  1
\end{pmatrix}
\]



\end{document}