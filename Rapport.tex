\documentclass[a4paper,11pt]{article}

% Packages
\usepackage[utf8]{inputenc}
\usepackage[T1]{fontenc}
\usepackage[french]{babel}
\usepackage{fancyhdr}
\usepackage{geometry}
\usepackage{amsmath}
\usepackage{amssymb}
\usepackage{algorithm}
\usepackage{algpseudocode}

% Page layout
\geometry{a4paper, margin=2.5cm}

% Fix header height issue
\setlength{\headheight}{14pt}

% Header and footer setup
\pagestyle{fancy}
\fancyhf{} % Clear all header and footer fields

% Header
\fancyhead[R]{ISIMA - Théorie Des Jeux} % Right header

% Footer - with left, center, and right elements
\fancyfoot[L]{Erkin Tunc Boya} % Left footer
\fancyfoot[C]{Université de Clermont Auvergne} % Center footer
\fancyfoot[R]{\thepage} % Right footer

% Optional header/footer lines
\renewcommand{\headrulewidth}{0.4pt} % Header line (set to 0pt to remove)
\renewcommand{\footrulewidth}{0.4pt} % Footer line (set to 0pt to remove)

\begin{document}

\begin{center}
\Huge{ISIMA - Théorie Des Jeux}\\[0.5cm]
\LARGE{TP1 de Théorie Des Jeux}\\[0.2cm]
\Large{Decembre 2025}\\[0.1cm]
\Large{Erkin Tunc BOYA}
\end{center}

\section{Ex1}
Il y a deux joueurs $A$ et $B$. Chacun possède trois jetons : un rouge, un blanc et un bleu. 
C'est un jeu à deux joueurs, à somme nulle : si la valeur est $> 0$, c'est un gain pour $A$ et une perte du même montant pour $B$, et réciproquement. Pour chaque issue, la somme des gains des deux joueurs est donc toujours égale à $0$. 
Pour commencer, les joueurs sélectionnent chacun un de leurs jetons et les exhibent simultanément. \\
 
Ils déterminent alors leur gain :


\begin{table}[h!]
    \centering
    \begin{tabular}{|c|c|}
        \hline
        \textbf{Lien entre couleurs} & \textbf{gain} \\ \hline
        rouge bat blanc              & 50  \\ 
        blanc bat bleu               & 40  \\ 
        bleu bat rouge               & 30  \\ 
        \text{même couleur}          & 0   \\ \hline
    \end{tabular}
\end{table}


\subsection{Modélisation du jeu}

On considère les stratégies pures suivantes pour chaque joueur (ordre dans lequel il utilise ses jetons) :

\subsubsection{Les stratégies}
\noindent
1 = (\text{rouge}, \text{blanc}, \text{bleu})\\
2 = (\text{rouge}, \text{bleu}, \text{blanc})\\
3 = (\text{blanc}, \text{rouge}, \text{bleu})\\
4 = (\text{blanc}, \text{bleu}, \text{rouge})\\
5 = (\text{bleu}, \text{rouge}, \text{blanc})\\
6 = (\text{bleu}, \text{blanc}, \text{rouge})\\[0.3cm]


La matrice de gains (pour le joueur $A$) est alors :
\[
\begin{array}{c|cccccc}
      & 1 & 2 & 3 & 4 & 5 & 6 \\ \hline
1 &  0   &  0   &   0   &  120  & -120 &   0   \\
2 &  0   &  0   &  120  &   0   &   0  & -120  \\
3 &  0   & -120 &   0   &   0   &   0  &  120  \\
4 & -120 &  0   &   0   &   0   &  120 &   0   \\
5 &  120 &  0   &   0   & -120  &   0  &   0   \\
6 &  0   & 120  & -120  &   0   &   0  &   0
\end{array}
\]

\bigskip

Exemple : case \(i = 1\), \(j = 4\).

La stratégie \(1\) du joueur~A est \((\text{rouge}, \text{blanc}, \text{bleu})\),  
la stratégie \(4\) du joueur~B est \((\text{blanc}, \text{bleu}, \text{rouge})\).

\begin{itemize}
    \item Coup 1 : rouge (A) contre blanc (B) → rouge bat blanc → \(+50\).
    \item Coup 2 : blanc (A) contre bleu (B) → blanc bat bleu → \(+40\).
    \item Coup 3 : bleu (A) contre rouge (B) → bleu bat rouge → \(+30\).
\end{itemize}

Ainsi, le gain total du joueur~A est
\[
50 + 40 + 30 = 120,
\]
d’où \(a_{1,4} = 120\).


\subsection{Stratégie pure}

On calcule d'abord les bornes classiques :
\[
V^- = \max_i \min_j a_{ij}, 
\qquad 
V^+ = \min_j \max_i a_{ij}.
\]

Pour chaque stratégie pure $i$ de $A$, le joueur $B$ peut toujours choisir une colonne qui donne à $A$ une perte de $-120$. Ainsi, pour chaque ligne $i$, on a 
\[
\min_j a_{ij} = -120.
\]
On en déduit :
\[
V^- = \max_i \min_j a_{ij} = -120.
\]

De même, pour chaque stratégie pure $j$ de $B$, le joueur $A$ peut choisir une ligne qui lui garantit un gain de $120$. Par conséquent,
\[
\max_i a_{ij} = 120 \quad \text{pour chaque colonne } j.
\]
D'où :
\[
V^+ = \min_j \max_i a_{ij} = 120.
\]

Ainsi,
\[
V^- \neq V^+ \qquad \Longrightarrow \qquad 
\text{pas de point selle et pas de solution en stratégies pures.}
\]


\subsection{Stratégies mixtes}

Une stratégie mixte est un vecteur de probabilités.

Pour le joueur~A :
\[
X = (x_1, \ldots, x_6), \qquad 
x_i \ge 0 \ \text{pour tout } i = 1,\ldots,6, \qquad
\sum_{i=1}^{6} x_i = 1.
\]

Pour le joueur~B :
\[
Y = (y_1, \ldots, y_6), \qquad
y_j \ge 0 \ \text{pour tout } j = 1,\ldots,6, \qquad
\sum_{j=1}^{6} y_j = 1.
\]

\bigskip

Dans notre modélisation, il n'existe pas de point selle puisque
\[
V^- \neq V^+.
\]

Nous cherchons donc des stratégies mixtes \(X^*\) et \(Y^*\)
telles que l'espérance de gain du joueur~A soit maximisée et celle de B minimisée.

\medskip

\paragraph{Programme du joueur~A (maximiseur, en ligne).}
On cherche une stratégie mixte 
\[
X = (x_1,\ldots,x_6)
\]
et une valeur garantie \(g\). Pour que \(X\) soit acceptable pour le joueur~A, il faut que,
contre chaque stratégie pure \(j\) du joueur~B, l’espérance de gain de \(A\) soit au moins égale
à \(g\) :
\[
E(X,j) = \sum_{i=1}^{6} a_{ij} x_i \ge g.
\]

En utilisant successivement les colonnes de la matrice de gains, on obtient les six contraintes
suivantes :
\[
\begin{cases}
-120\,x_4 + 120\,x_5 \ge g, \\[0.2em]
-120\,x_3 + 120\,x_6 \ge g, \\[0.2em]
120\,x_2 - 120\,x_6 \ge g, \\[0.2em]
120\,x_1 - 120\,x_5 \ge g, \\[0.2em]
-120\,x_1 + 120\,x_4 \ge g, \\[0.2em]
-120\,x_2 + 120\,x_3 \ge g, \\[0.4em]
x_1 + x_2 + x_3 + x_4 + x_5 + x_6 = 1, \\[0.2em]
x_i \ge 0,\quad i=1,\ldots,6.
\end{cases}
\]

L’objectif du joueur~A est
\[
\max g
\]
sous ces contraintes.

\medskip

La résolution (par exemple via OPL) donne :
\[
g^{*} = 0,
\qquad
x_1^{*} = x_4^{*} = x_5^{*} = \frac{1}{3},
\qquad
x_2^{*} = x_3^{*} = x_6^{*} = 0.
\]

Ainsi, la stratégie optimale du joueur~A est
\[
x^{*} = \left( 
\frac{1}{3},\;
0,\;
0,\;
\frac{1}{3},\;
\frac{1}{3},\;
0
\right)^{T},
\qquad
v = g^{*} = 0.
\]

\paragraph{Vérification.}
Pour cette stratégie \(X^{*}\), on vérifie que, pour chaque colonne \(j\),
\[
E(X^{*},j)=0=g^{*}.
\]
Chaque contrainte est donc satisfaite ou saturée, ce qui confirme que \(X^{*}\) est optimale.


\paragraph{Programme du joueur~B (minimiseur, en colonne).}
Le joueur~B cherche une stratégie mixte 
\[
Y = (y_1,\ldots,y_6)
\]
et une valeur \(h\) telles que, pour chaque stratégie pure \(i\) de \(A\),
l’espérance de gain de \(A\) ne dépasse pas \(h\) :
\[
E(i,Y) = \sum_{j=1}^{6} a_{ij} y_j \le h.
\]

En utilisant les lignes de la matrice de gains, on obtient les six contraintes :
\[
\begin{cases}
120\,y_4 - 120\,y_5 \le h,\\[0.2em]
120\,y_3 - 120\,y_6 \le h,\\[0.2em]
-120\,y_2 + 120\,y_6 \le h,\\[0.2em]
-120\,y_1 + 120\,y_5 \le h,\\[0.2em]
120\,y_1 - 120\,y_4 \le h,\\[0.2em]
120\,y_2 - 120\,y_3 \le h,\\[0.4em]
y_1 + y_2 + y_3 + y_4 + y_5 + y_6 = 1,\\[0.2em]
y_j \ge 0,\quad j=1,\ldots,6.
\end{cases}
\]

L’objectif du joueur~B est :
\[
\min h.
\]

\medskip

En résolvant ce programme linéaire, on obtient :
\[
h^{*}=0.
\]

Une solution optimale pour le joueur~B est :
\[
y^{*}_{1}=y^{*}_{4}=y^{*}_{5}=\frac{1}{3},
\qquad
y^{*}_{2}=y^{*}_{3}=y^{*}_{6}=0.
\]

\paragraph{Justification.}
On vérifie que, pour chaque ligne \(i\),
\[
E(i,Y^{*}) = 0 = h^{*}.
\]
Toutes les contraintes sont satisfaites ou saturées, ce qui confirme que \(Y^{*}\) est optimale.
De plus, la matrice du jeu est antisymétrique (chaque gain non nul apparaît avec son opposé),
ce qui explique que les stratégies optimales de \(A\) et \(B\) coïncident.


\paragraph{Question (c).}
Dans l’énoncé, on suppose que le joueur maximisateur utilise aussi souvent les trois
combinaisons suivantes :
\begin{itemize}
    \item[(i)] Bleu – Rouge – Blanc,
    \item[(ii)] Blanc – Bleu – Rouge,
    \item[(iii)] Rouge – Blanc – Bleu.
\end{itemize}

Dans notre numérotation des stratégies pures, ces combinaisons correspondent
respectivement aux stratégies \(5\), \(4\) et \(1\).
Ainsi, jouer chacune de ces trois stratégies avec probabilité \(1/3\) revient à jouer
la stratégie mixte :
\[
X = \left( \frac{1}{3},\, 0,\, 0,\, \frac{1}{3},\, \frac{1}{3},\, 0 \right).
\]

Or, cette stratégie est exactement la solution optimale obtenue dans le programme
linéaire du joueur~A :
\[
x^{*}_1 = x^{*}_4 = x^{*}_5 = \frac{1}{3}, 
\qquad
x^{*}_2 = x^{*}_3 = x^{*}_6 = 0.
\]

De plus, pour cette stratégie, l’espérance de gain contre chacune des stratégies pures
du joueur~B vaut :
\[
E(X,j) = 0, \qquad \forall j=1,\ldots,6.
\]
C’est donc bien la meilleure valeur que le joueur~A peut garantir, c’est-à-dire
\[
v = 0.
\]

Ainsi, le fait de jouer les trois combinaisons proposées avec la même fréquence assure
au joueur~A un gain espéré maximal, égal à la valeur du jeu.


%------------------------------------------------------


\section{Ex2 Jeu de Domino}

Le joueur $X$ choisit une configuration de dominos parmi les $7$ possibles sur une grille $2\times 3$ (cases blanches). 
Le joueur $Y$ choisit simultanément une case parmi les $6$ cases de la grille.

Si la configuration choisie par $X$ recouvre la case annoncée par $Y$, alors $Y$ gagne ; sinon, c'est $X$ qui gagne.

C'est un jeu à somme nulle : si la valeur est $> 0$, c'est un gain pour $X$ et une perte du même montant pour $Y$, et réciproquement. 
On code par exemple :
\[
\text{gain de }X =
\begin{cases}
+1 & \text{si $X$ gagne},\\
-1 & \text{si $Y$ gagne}.
\end{cases}
\]

\subsection{Modélisation}

Il y a $7$ configurations différentes possibles pour $X$, donc $7$ lignes dans la matrice de jeu.
Les $6$ colonnes correspondent aux $6$ cases que le joueur $Y$ peut annoncer (numérotées de $1$ à $6$).

Pour chaque configuration $i$ et chaque case $j$, on définit le gain de $X$ par
\[
a_{ij} =
\begin{cases}
1 & \text{si la configuration $i$ recouvre la case $j$},\\
-1 & \text{sinon.}
\end{cases}
\]

Cela conduit à la matrice de gains suivante (pour le joueur $X$) :
\[
A =
\begin{pmatrix}
-1 &  1 &  1 & -1 &  1 &  1 \\
 1 &  1 & -1 &  1 &  1 & -1 \\
 1 & -1 &  1 &  1 &  1 &  1 \\
 1 & -1 & -1 &  1 &  1 &  1 \\
-1 & -1 &  1 &  1 &  1 &  1 \\
 1 &  1 &  1 &  1 & -1 & -1 \\
 1 &  1 &  1 & -1 & -1 &  1
\end{pmatrix}.
\]


\subsection{Stratégie pure}

On calcule d’abord les bornes classiques :
\[
V^- = \max_i \min_j a_{ij},
\qquad
V^+ = \min_j \max_i a_{ij}.
\]

Pour chaque ligne du joueur \(X\), on remarque que la configuration choisie ne
recouvre pas au moins une case proposée par \(Y\), ce qui donne toujours une
issue défavorable de \(-1\). Ainsi :
\[
\min_j a_{ij} = -1 \qquad \text{pour chaque } i=1,\ldots,7.
\]
D’où :
\[
V^- = \max(-1,-1,-1,-1,-1,-1,-1) = -1.
\]

De même, pour chaque colonne \(j\), il existe toujours au moins une configuration
de \(X\) qui recouvre cette case, ce qui donne un gain de \(+1\). Autrement dit :
\[
\max_i a_{ij} = 1 \qquad \text{pour chaque } j=1,\ldots,6.
\]
On en déduit :
\[
V^+ = \min(1,1,1,1,1,1) = 1.
\]

On a donc :
\[
V^- \neq V^+ \qquad\Longrightarrow\qquad
\text{pas de point selle et pas de solution en stratégies pures}.
\]

\subsection{Stratégies mixtes et résolution du jeu}

Comme il n’existe pas de point selle en stratégies pures, on étudie le jeu en stratégies mixtes.

Le joueur $X$ choisit une distribution
\[
X = (x_1,\ldots,x_7), \qquad x_i \ge 0,\;\; \sum_{i=1}^7 x_i = 1,
\]
et le joueur $Y$ choisit
\[
Y = (y_1,\ldots,y_6), \qquad y_j \ge 0,\;\; \sum_{j=1}^6 y_j = 1.
\]

L’espérance de gain pour $X$ est
\[
E(X,Y) = X A Y^{T}.
\]

Nous cherchons des stratégies mixtes optimales $X^{*}$ et $Y^{*}$ ainsi que la valeur du jeu $v$ telles que
\[
\max_X \min_Y E(X,Y) = \min_Y \max_X E(X,Y) = v.
\]

\subsubsection*{Résolution numérique}

En résolvant les programmes linéaires correspondants (par exemple avec OPL), on obtient la valeur du jeu :
\[
v = 0.
\]

Autrement dit, aucun des deux joueurs ne peut garantir un gain strictement positif en jouant de manière optimale.

Une solution optimale pour le joueur $X$ est :
\[
X^{*} =
\left(
0,\;
\frac{1}{6},\;
\frac{1}{6},\;
\frac{1}{6},\;
\frac{1}{6},\;
\frac{1}{6},\;
\frac{1}{6}
\right).
\]

Une solution optimale pour le joueur $Y$ est :
\[
Y^{*} =
\left(
\frac{1}{6},\;
\frac{1}{6},\;
\frac{1}{6},\;
\frac{1}{6},\;
\frac{1}{6},\;
\frac{1}{6}
\right).
\]

\subsubsection*{Interprétation}

Les deux joueurs jouent uniformément leurs stratégies disponibles (sauf la configuration 1 de $X$, qui est dominée).

Comme $v = 0$, aucun joueur n’est avantagé dans ce jeu : chacun peut garantir une espérance nulle, mais pas davantage.

Ainsi, les stratégies mixtes uniformes constituent des stratégies optimales pour les deux joueurs.

\section*{Exercice 3}

\subsection*{a) Les 3 stratégies possibles pour chaque entreprise}

\begin{enumerate}
    \item \textbf{Simultané (S)} : L’entreprise améliore les deux produits en même temps. Les deux seront prêts au bout de 12 mois.
    
    \item \textbf{Produit 1 puis produit 2 (P12)} : L’entreprise commence par améliorer le produit 1 qui sort au bout de 9 ou 10 mois, puis le produit 2 \textbf{9 mois après le premier}.
    
    \item \textbf{Produit 2 puis produit 1 (P21)} : L’entreprise commence par améliorer le produit 2 qui sort au bout de 9 ou 10 mois, puis le produit 1 \textbf{9 mois après le premier}.
\end{enumerate}
\subsection*{b) Matrices des gains par produit}

En utilisant les règles de l'énoncé :

\begin{itemize}
    \item Gain de 8 \% si A et B sortent simultanément.
    \item Gain de 20, 30, 40 \% si A est en avance de 2, 6 ou 8 mois.
    \item Perte de 4, 10, 12, 14 \% si B est en avance de 1, 3, 7 ou 10 mois.
\end{itemize}

On peut alors construire les deux matrices de gains pour l’entreprise A, une pour chaque produit, en fonction des stratégies choisies par A et B (S, P12, P21).


\begin{table}[h!]
\subsubsection*{Produit 1 – Gains de A}
\centering
\begin{tabular}{c|c|c|c}
\textbf{Stratégie A \ Stratégie B} & S & P12 & P21 \\ \hline
S   & 0,08  & 0,20  & 0,30 \\
P12 & 0,20  & -0,04 & 0,40 \\
P21 & -0,12 & -0,14 & -0,04
\end{tabular}
\caption{Matrice des gains de l’entreprise A pour le Produit 1}
\label{tab:produit1}
\end{table}



\begin{table}[h!]
\subsubsection*{Produit 2 – Gains de A}

\centering
\begin{tabular}{c|c|c|c}
\textbf{Stratégie A \ Stratégie B} & S & P12 & P21 \\ \hline
S   & 0,08  & 0,30  & -0,10 \\
P12 & -0,12 & -0,04 & -0,14 \\
P21 & 0,20  & 0,40  & -0,04
\end{tabular}
\caption{Matrice des gains de l’entreprise A pour le Produit 2}
\end{table}
 

\begin{quote}
En s’appuyant sur l’énoncé : Pour chaque type de produit, si les deux entreprises proposent leur modèle amélioré simultanément, A augmente son offre des ventes totales futures dans ce produit de 8\% (c’est-à-dire que sa part passe de 25\% à 33\%). De la même manière, A augmente son offre de 20, 30 et 40\% du total si le produit est disponible respectivement 2, 6 et 8 mois plus tôt que B. D’autre part, A perd 4, 10, 12 et 14\% du total si B propose le produit respectivement 1, 3, 7 et 10 mois plus tôt.
\end{quote}

On a donc deux entreprises, A et B, qui se partagent le marché.

Si l’entreprise A gagne une part, B la perd exactement et inversement.  
 

\begin{center}
\begin{tabular}{c|c|c|c}
\textbf{Stratégie A \ Stratégie B} & S & 12 & 21 \\ \hline
S & 0,08 $\cdot$ Q$_1$ + 0,08 $\cdot$ Q$_2$ & -0,10 $\cdot$ Q$_1$ + 0,30 $\cdot$ Q$_2$ & 0,30 $\cdot$ Q$_1$ - 0,10 $\cdot$ Q$_2$ \\
12  & 0,20 $\cdot$ Q$_1$ - 0,12 $\cdot$ Q$_2$ & -0,04 $\cdot$ (Q$_1$ + Q$_2$) & 0,40 $\cdot$ Q$_1$ - 0,14 $\cdot$ Q$_2$ \\
21  & -0,12 $\cdot$ Q$_1$ + 0,20 $\cdot$ Q$_2$ & -0,14 $\cdot$ Q$_1$ + 0,40 $\cdot$ Q$_2$ & -0,04 $\cdot$ (Q$_1$ + Q$_2$)
\end{tabular}
\end{center}
\subsection*{d) Existence d’un point selle}

Pour savoir si une stratégie stable existe pour A et B, on cherche un \textbf{point selle}.  
Un point selle correspond à une situation où aucun joueur ne peut améliorer son gain en changeant de stratégie, tant que l’autre reste sur la sienne.

\subsubsection*{Cas 1 : \(Q_1 = Q_2 = x\)}

Si les deux produits ont la même quantité, on remplace \(Q_1\) et \(Q_2\) par \(x\) dans la matrice des gains de l’entreprise A :

\begin{table}[h!]
\centering
\begin{tabular}{c|c|c|c}
\textbf{Stratégie A \ Stratégie B} & S & 12 & 21 \\ \hline
S  & 0,16x  & 0,20x  & 0,20x \\
12 & 0,08x  & -0,08x & 0,26x \\
21 & 0,08x  & 0,26x  & -0,08x
\end{tabular}
\caption{Matrice des gains de l’entreprise A pour \(Q_1 = Q_2 = x\)}
\end{table}

Les bornes du jeu sont :  
\[
V^+ = \min_j \max_i a_{ij} = 0,16x, \quad
V^- = \max_i \min_j a_{ij} = 0,16x
\]

Comme \(V^+ = V^-\), il existe un \textbf{point selle}.  
La stratégie stable est que \textbf{les deux entreprises choisissent S (simultané)}.

\subsubsection*{Cas 2 : \(Q_1 = Q_2 / 2 = x\)}

Si le produit 1 représente la moitié du produit 2, on remplace \(Q_1 = x\) et \(Q_2 = 2x\). La matrice devient :

\begin{table}[h!]
\centering
\begin{tabular}{c|c|c|c}
\textbf{Stratégie A \ Stratégie B} & S & 12 & 21 \\ \hline
S  & 0,12x  & 0,05x  & 0,25x \\
12 & 0,14x  & -0,06x & 0,33x \\
21 & -0,02x & 0,06x  & -0,06x
\end{tabular}
\caption{Matrice des gains de l’entreprise A pour \(Q_1 = Q_2/2 = x\)}
\end{table}

Calcul des bornes :  
\[
V^+ = \min_j \max_i a_{ij} = 0,06x, \quad
V^- = \max_i \min_j a_{ij} = 0,05x
\]

Ici \(V^+ \neq V^-\), donc \textbf{il n’existe pas de point selle}.  
Aucune stratégie pure n’est stable dans ce cas.


\end{document}
