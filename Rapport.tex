\documentclass[a4paper,11pt]{article}

% Packages
\usepackage[utf8]{inputenc}
\usepackage[T1]{fontenc}
\usepackage[french]{babel}
\usepackage{fancyhdr}
\usepackage{geometry}
\usepackage{amsmath}
\usepackage{amssymb}
\usepackage{algorithm}
\usepackage{algpseudocode}

% Page layout
\geometry{a4paper, margin=2.5cm}

% Fix header height issue
\setlength{\headheight}{14pt}

% Header and footer setup
\pagestyle{fancy}
\fancyhf{} % Clear all header and footer fields

% Header
\fancyhead[R]{ISIMA - Théorie Des Jeux} % Right header

% Footer - with left, center, and right elements
\fancyfoot[L]{BOYA, HAFSOUNI} % Left footer
\fancyfoot[C]{Université de Clermont Auvergne} % Center footer
\fancyfoot[R]{\thepage} % Right footer

% Optional header/footer lines
\renewcommand{\headrulewidth}{0.4pt} % Header line (set to 0pt to remove)
\renewcommand{\footrulewidth}{0.4pt} % Footer line (set to 0pt to remove)

\begin{document}

\begin{center}
\Huge{ISIMA - Théorie Des Jeux}\\[0.5cm]
\LARGE{TP1 de Théorie Des Jeux}\\[0.2cm]
\Large{Decembre 2025}\\[0.1cm]
\Large{Erkin Tunc BOYA , Raed HAFSOUNI}
\end{center}

\section*{Exercice 1 -- Jeu des jetons colorés}

\subsection*{(a) Jeu à somme nulle}

À chaque coup, l’un des deux joueurs gagne et l’autre perd le même montant
(50, 40, 30 ou 0). Le gain total d’une partie est la somme des gains sur les
trois coups, et pour chaque issue la somme des gains de $A$ et $B$ est nulle.
Le jeu est donc à somme nulle.

\subsection*{(b) Matrice de gains}

Chaque joueur possède les trois jetons \textit{rouge, blanc, bleu} et doit
choisir un ordre d’utilisation de ces jetons sur trois coups. On numérote les
stratégies pures (permutations) comme suit :
\[
\begin{aligned}
1 &= (\text{rouge}, \text{blanc}, \text{bleu}),\\
2 &= (\text{rouge}, \text{bleu}, \text{blanc}),\\
3 &= (\text{blanc}, \text{rouge}, \text{bleu}),\\
4 &= (\text{blanc}, \text{bleu}, \text{rouge}),\\
5 &= (\text{bleu}, \text{rouge}, \text{blanc}),\\
6 &= (\text{bleu}, \text{blanc}, \text{rouge}).
\end{aligned}
\]

En appliquant les règles
(rouge bat blanc : $+50$, blanc bat bleu : $+40$, bleu bat rouge : $+30$,
même couleur : $0$),
on obtient la matrice de gain (pour $A$) :

\[
A =
\begin{array}{c|cccccc}
      & 1 & 2 & 3 & 4 & 5 & 6 \\ \hline
1 &   0  &   0  &   0  & 120 & -120 &   0  \\
2 &   0  &   0  & 120  &  0  &   0  & -120 \\
3 &   0  & -120 &   0  &  0  &   0  & 120  \\
4 & -120 &   0  &   0  &  0  & 120  &  0   \\
5 & 120  &   0  &   0  & -120&   0  &  0   \\
6 &  0   & 120  & -120 &  0  &   0  &  0
\end{array}
\]

Pour chaque ligne $i$, le joueur $B$ peut forcer un résultat $-120$ pour $A$,
donc
\[
\min_j a_{ij} = -120 \quad \Rightarrow \quad V^- = \max_i \min_j a_{ij} = -120.
\]

Pour chaque colonne $j$, le joueur $A$ peut forcer un résultat $+120$,
donc
\[
\max_i a_{ij} = 120 \quad \Rightarrow \quad V^+ = \min_j \max_i a_{ij} = 120.
\]

Comme
\[
V^- \neq V^+,
\]
il n’y a pas de point selle et pas de solution en stratégies pures.

\subsection*{(c) Stratégies mixtes et résolution par PL}

On introduit une stratégie mixte pour $A$ :
\[
X = (x_1,\ldots,x_6), \qquad x_i \ge 0,\quad \sum_{i=1}^{6} x_i = 1,
\]
et une valeur garantie $g$. Le programme linéaire du joueur $A$ est :
\[
\max g
\]
sous les contraintes
\[
E(X,j) = \sum_{i=1}^{6} a_{ij} x_i \;\ge\; g, \quad j=1,\ldots,6,
\]
et
\[
\sum_{i=1}^{6} x_i = 1,\quad x_i \ge 0.
\]

La résolution numérique avec OPL donne
\[
g^{*} = 0,
\qquad
x_1^{*} = x_4^{*} = x_5^{*} = \frac{1}{3},
\qquad
x_2^{*} = x_3^{*} = x_6^{*} = 0.
\]

Donc la stratégie optimale de $A$ est
\[
X^{*} =
\left(
\frac{1}{3},\;
0,\;
0,\;
\frac{1}{3},\;
\frac{1}{3},\;
0
\right),
\qquad
v = 0.
\]

Par symétrie et par résolution analogue du programme du joueur $B$, on obtient
une stratégie optimale de même forme pour $B$, et la valeur du jeu est
également $0$ pour lui (jeu équilibré).

\subsection*{(d) Lien avec les trois combinaisons proposées}

Dans l’énoncé, on propose au joueur maximisateur de jouer aussi souvent les
trois combinaisons suivantes :
\begin{itemize}
    \item Bleu – Rouge – Blanc,
    \item Blanc – Bleu – Rouge,
    \item Rouge – Blanc – Bleu.
\end{itemize}

Dans notre numérotation, cela correspond exactement aux stratégies
\[
5 = (\text{bleu}, \text{rouge}, \text{blanc}), \quad
4 = (\text{blanc}, \text{bleu}, \text{rouge}), \quad
1 = (\text{rouge}, \text{blanc}, \text{bleu}).
\]

Jouer chacune de ces trois stratégies avec probabilité $\frac{1}{3}$ revient donc
à utiliser la stratégie mixte
\[
X = \left(
\frac{1}{3},\;
0,\;
0,\;
\frac{1}{3},\;
\frac{1}{3},\;
0
\right),
\]
qui est exactement la solution optimale $X^{*}$ trouvée par le programme linéaire.

Pour cette stratégie, l’espérance de gain de $A$ contre chaque stratégie pure
de $B$ est nulle :
\[
E(X,j) = 0,\quad \forall j=1,\ldots,6,
\]
et le joueur $A$ garantit donc la valeur du jeu
\[
v = 0.
\]


%------------------------------------------------------


\section*{Exercice 2 -- Jeu de Domino}

\subsection*{(a) Jeu à somme nulle}

On considère une grille $2\times 3$ (6 cases blanches) et les 7 configurations possibles
d’un domino couvrant deux cases. Le joueur $X$ choisit une configuration, le joueur $Y$
choisit simultanément une case parmi les 6.

\begin{itemize}
  \item Si la case choisie par $Y$ est recouverte par la configuration de $X$, alors $Y$ gagne.
  \item Sinon, $X$ gagne.
\end{itemize}

On code le gain de $X$ par
\[
\text{gain}(X) =
\begin{cases}
+1 & \text{si $X$ gagne},\\[1mm]
-1 & \text{si $Y$ gagne}.
\end{cases}
\]
Dans tous les cas, le gain de $Y$ vaut $-\text{gain}(X)$, donc la somme des gains est toujours nulle.
Le jeu est donc bien un jeu à somme nulle.

\subsection*{(b) Matrice de gains}

Les 7 configurations possibles de placement du domino (numérotées de 1 à 7) sont les stratégies
pures de $X$. Les 6 cases de la grille (numérotées de 1 à 6) sont les stratégies pures de $Y$.

On note $a_{ij}$ le gain de $X$ lorsque $X$ joue la configuration $i$ et $Y$ annonce la case $j$.
On obtient la matrice de gains suivante (gain pour $X$) :
\[
A =
\begin{pmatrix}
-1 &  1 &  1 & -1 &  1 &  1 \\
 1 &  1 & -1 &  1 &  1 & -1 \\
 1 & -1 &  1 &  1 & -1 &  1 \\
 1 & -1 & -1 &  1 &  1 &  1 \\
-1 & -1 &  1 &  1 &  1 &  1 \\
 1 &  1 &  1 &  1 & -1 & -1 \\
 1 &  1 &  1 & -1 & -1 &  1
\end{pmatrix}.
\]

Pour chaque stratégie pure $i$ de $X$, $Y$ peut toujours choisir une colonne qui donne $-1$ à $X$,
donc
\[
\min_j a_{ij} = -1 \quad \Rightarrow \quad
V^- = \max_i \min_j a_{ij} = -1.
\]

Pour chaque stratégie pure $j$ de $Y$, $X$ peut choisir une ligne avec un gain $+1$, donc
\[
\max_i a_{ij} = 1 \quad \Rightarrow \quad
V^+ = \min_j \max_i a_{ij} = 1.
\]

Comme
\[
V^- \neq V^+,
\]
il n’y a pas de point selle en stratégies pures, donc pas de solution en stratégies pures.

\subsection*{(c) Stratégies mixtes et résolution par PL}

On introduit une stratégie mixte pour $X$ :
\[
X = (x_1,\ldots,x_7), \qquad
x_i \ge 0,\quad \sum_{i=1}^{7} x_i = 1.
\]

L’espérance de gain de $X$ contre la stratégie pure $j$ de $Y$ est
\[
E(X,j) = \sum_{i=1}^{7} a_{ij} x_i.
\]
Pour que $X$ se garantisse au moins un gain $g$, il faut
\[
E(X,j) \ge g, \qquad j = 1,\ldots,6.
\]

Le programme linéaire du joueur $X$ est donc :
\[
\max g
\]
sous les contraintes
\[
\sum_{i=1}^{7} a_{ij} x_i \;\ge\; g, \qquad j=1,\ldots,6,
\]
\[
\sum_{i=1}^{7} x_i = 1, \qquad x_i \ge 0.
\]

Ce modèle est exactement celui codé dans le fichier \texttt{Ex2.txt} (une contrainte par colonne
de la matrice $A$).

\medskip

La résolution numérique avec OPL donne :
\[
g^{*} = \frac{1}{3} \approx 0.3333,
\]
et la solution optimale
\[
x_1^{*} = x_4^{*} = x_6^{*} = \frac{1}{3}, \qquad
x_2^{*} = x_3^{*} = x_5^{*} = x_7^{*} = 0.
\]

Ainsi, une stratégie optimale pour le joueur $X$ est
\[
X^{*} =
\left(
\frac{1}{3},\;
0,\;
0,\;
\frac{1}{3},\;
0,\;
\frac{1}{3},\;
0
\right).
\]

Dans cette stratégie, seules les configurations $1$, $4$ et $6$ sont utilisées, chacune avec
probabilité $1/3$. On vérifie que, pour chaque case $j$ choisie par $Y$,
\[
E(X^{*},j) = \frac{1}{3},
\]
donc $X$ se garantit un gain espéré de $1/3$ quelle que soit la stratégie pure de $Y$.

Par symétrie et via le programme linéaire dual pour $Y$, on obtient une stratégie mixte
optimale $Y^{*}$ telle que la valeur du jeu soit également $1/3$ pour $X$.

\subsection*{Conclusion : valeur du jeu et joueur avantagé}

La valeur du jeu (gain pour $X$) est
\[
v = g^{*} = \frac{1}{3} > 0.
\]

Le joueur $X$ (qui choisit la configuration de dominos) est donc avantagé : il peut garantir un
gain espéré strictement positif, tandis que $Y$ ne peut faire mieux que limiter ce gain à $1/3$.




\section*{Exercice 3}

\subsection*{a) Les 3 stratégies possibles pour chaque entreprise}

\begin{enumerate}
    \item \textbf{Simultané (S)} : L’entreprise améliore les deux produits en même temps. Les deux seront prêts au bout de 12 mois.
    
    \item \textbf{Produit 1 puis produit 2 (P12)} : L’entreprise commence par améliorer le produit 1 qui sort au bout de 9 ou 10 mois, puis le produit 2 \textbf{9 mois après le premier}.
    
    \item \textbf{Produit 2 puis produit 1 (P21)} : L’entreprise commence par améliorer le produit 2 qui sort au bout de 9 ou 10 mois, puis le produit 1 \textbf{9 mois après le premier}.
\end{enumerate}
\subsection*{b) Matrices des gains par produit}

En utilisant les règles de l'énoncé :

\begin{itemize}
    \item Gain de 8 \% si A et B sortent simultanément.
    \item Gain de 20, 30, 40 \% si A est en avance de 2, 6 ou 8 mois.
    \item Perte de 4, 10, 12, 14 \% si B est en avance de 1, 3, 7 ou 10 mois.
\end{itemize}

On peut alors construire les deux matrices de gains pour l’entreprise A, une pour chaque produit, en fonction des stratégies choisies par A et B (S, P12, P21).


\begin{table}[h!]
\subsubsection*{Produit 1 – Gains de A}
\centering
\begin{tabular}{c|c|c|c}
\textbf{Stratégie A \ Stratégie B} & S & P12 & P21 \\ \hline
S   & 0,08  & 0,20  & 0,30 \\
P12 & 0,20  & -0,04 & 0,40 \\
P21 & -0,12 & -0,14 & -0,04
\end{tabular}
\caption{Matrice des gains de l’entreprise A pour le Produit 1}
\label{tab:produit1}
\end{table}



\begin{table}[h!]
\subsubsection*{Produit 2 – Gains de A}

\centering
\begin{tabular}{c|c|c|c}
\textbf{Stratégie A \ Stratégie B} & S & P12 & P21 \\ \hline
S   & 0,08  & 0,30  & -0,10 \\
P12 & -0,12 & -0,04 & -0,14 \\
P21 & 0,20  & 0,40  & -0,04
\end{tabular}
\caption{Matrice des gains de l’entreprise A pour le Produit 2}
\end{table}
 

\begin{quote}
En s’appuyant sur l’énoncé : Pour chaque type de produit, si les deux entreprises proposent leur modèle amélioré simultanément, A augmente son offre des ventes totales futures dans ce produit de 8\% (c’est-à-dire que sa part passe de 25\% à 33\%). De la même manière, A augmente son offre de 20, 30 et 40\% du total si le produit est disponible respectivement 2, 6 et 8 mois plus tôt que B. D’autre part, A perd 4, 10, 12 et 14\% du total si B propose le produit respectivement 1, 3, 7 et 10 mois plus tôt.
\end{quote}

On a donc deux entreprises, A et B, qui se partagent le marché.

Si l’entreprise A gagne une part, B la perd exactement et inversement.  
 

\begin{center}
\begin{tabular}{c|c|c|c}
\textbf{Stratégie A \ Stratégie B} & S & 12 & 21 \\ \hline
S & 0,08 $\cdot$ Q$_1$ + 0,08 $\cdot$ Q$_2$ & -0,10 $\cdot$ Q$_1$ + 0,30 $\cdot$ Q$_2$ & 0,30 $\cdot$ Q$_1$ - 0,10 $\cdot$ Q$_2$ \\
12  & 0,20 $\cdot$ Q$_1$ - 0,12 $\cdot$ Q$_2$ & -0,04 $\cdot$ (Q$_1$ + Q$_2$) & 0,40 $\cdot$ Q$_1$ - 0,14 $\cdot$ Q$_2$ \\
21  & -0,12 $\cdot$ Q$_1$ + 0,20 $\cdot$ Q$_2$ & -0,14 $\cdot$ Q$_1$ + 0,40 $\cdot$ Q$_2$ & -0,04 $\cdot$ (Q$_1$ + Q$_2$)
\end{tabular}
\end{center}
\subsection*{d) Existence d’un point selle}

Pour savoir si une stratégie stable existe pour A et B, on cherche un \textbf{point selle}.  
Un point selle correspond à une situation où aucun joueur ne peut améliorer son gain en changeant de stratégie, tant que l’autre reste sur la sienne.

\subsubsection*{Cas 1 : \(Q_1 = Q_2 = x\)}

Si les deux produits ont la même quantité, on remplace \(Q_1\) et \(Q_2\) par \(x\) dans la matrice des gains de l’entreprise A :

\begin{table}[h!]
\centering
\begin{tabular}{c|c|c|c}
\textbf{Stratégie A \ Stratégie B} & S & 12 & 21 \\ \hline
S  & 0,16x  & 0,20x  & 0,20x \\
12 & 0,08x  & -0,08x & 0,26x \\
21 & 0,08x  & 0,26x  & -0,08x
\end{tabular}
\caption{Matrice des gains de l’entreprise A pour \(Q_1 = Q_2 = x\)}
\end{table}

Les bornes du jeu sont :  
\[
V^+ = \min_j \max_i a_{ij} = 0,16x, \quad
V^- = \max_i \min_j a_{ij} = 0,16x
\]

Comme \(V^+ = V^-\), il existe un \textbf{point selle}.  
La stratégie stable est que \textbf{les deux entreprises choisissent S (simultané)}.

\subsubsection*{Cas 2 : \(Q_1 = Q_2 / 2 = x\)}

Si le produit 1 représente la moitié du produit 2, on remplace \(Q_1 = x\) et \(Q_2 = 2x\). La matrice devient :

\begin{table}[h!]
\centering
\begin{tabular}{c|c|c|c}
\textbf{Stratégie A \ Stratégie B} & S & 12 & 21 \\ \hline
S  & 0,12x  & 0,05x  & 0,25x \\
12 & 0,14x  & -0,06x & 0,33x \\
21 & -0,02x & 0,06x  & -0,06x
\end{tabular}
\caption{Matrice des gains de l’entreprise A pour \(Q_1 = Q_2/2 = x\)}
\end{table}

Calcul des bornes :  
\[
V^+ = \min_j \max_i a_{ij} = 0,06x, \quad
V^- = \max_i \min_j a_{ij} = 0,05x
\]

Ici \(V^+ \neq V^-\), donc \textbf{il n’existe pas de point selle}.  
Aucune stratégie pure n’est stable dans ce cas.


\end{document}
