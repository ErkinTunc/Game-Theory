\documentclass[a4paper,11pt]{article}

% Packages
\usepackage[utf8]{inputenc}
\usepackage[T1]{fontenc}
\usepackage[french]{babel}
\usepackage{fancyhdr}
\usepackage{geometry}
\usepackage{amsmath}
\usepackage{amssymb}
\usepackage{algorithm}
\usepackage{algpseudocode}

% Page layout
\geometry{a4paper, margin=2.5cm}

% Fix header height issue
\setlength{\headheight}{14pt}

% Header and footer setup
\pagestyle{fancy}
\fancyhf{} % Clear all header and footer fields

% Header
\fancyhead[R]{ISIMA - Théorie Des Jeux} % Right header

% Footer - with left, center, and right elements
\fancyfoot[L]{Erkin Tunc Boya} % Left footer
\fancyfoot[C]{Université de Clermont Auvergne} % Center footer
\fancyfoot[R]{\thepage} % Right footer

% Optional header/footer lines
\renewcommand{\headrulewidth}{0.4pt} % Header line (set to 0pt to remove)
\renewcommand{\footrulewidth}{0.4pt} % Footer line (set to 0pt to remove)

\begin{document}

\begin{center}
\Huge{ISIMA - Théorie Des Jeux}\\[0.5cm]
\LARGE{TP1 de Théorie Des Jeux}\\[0.2cm]
\Large{Decembre 2025}\\[0.1cm]
\Large{Erkin Tunc BOYA}
\end{center}

\section{Ex1}
Il y a 2 joueurs A et B .Chacun possede 3 jetons : un rouge, un blanc et un bleu. C'est un jeu de 2 jouers, Jeux à somme nulle. Pour commencer les joueurs sélectionnent chacun un de leur jeton et les exhibent simultanément. Ils d´eterminent alors leur
gain

 

\begin{table}[h!]
    \centering
    \begin{tabular}{|c|c|}
        \hline
        \textbf{Lien entre couleurs} & \textbf{gain} \\ \hline
        rouge bat blanc              & 50		\\ 
        blanc bat bleu               & 40		\\ 
        bleu bat rouge               & 30		\\ 
        \text{même couleur}          & 0		\\ \hline
    \end{tabular}
\end{table}


\subsection{Modeliastion de Jeu}
\[
\begin{array}{c|cccccc}
      & 1 & 2 & 3 & 4 & 5 & 6 \\
\hline
1 &  0   &  0   &   0   &  120  & -120 &   0   \\
2 &  0   &  0   &  120  &   0   &   0  & -120  \\
3 &  0   & -120 &   0   &   0   &   0  &  120  \\
4 & -120 &  0   &   0   &   0   &  120 &   0   \\
5 &  120 &  0   &   0   & -120  &   0  &   0   \\
6 &  0   & 120  & -120  &   0   &   0  &   0
\end{array}
\]

\subsubsection{Les straregies}
\noindent 1 = (\text{rouge}, \text{blanc}, \text{bleu})\\
2 = (\text{rouge}, \text{bleu}, \text{blanc})\\
3 = (\text{blanc}, \text{rouge}, \text{bleu})\\
4 = (\text{blanc}, \text{bleu}, \text{rouge})\\
5 = (\text{bleu}, \text{rouge}, \text{blanc})\\
6 = (\text{bleu}, \text{blanc}, \text{rouge})\\


\section{Ex2}

\subsection{Modelisation}
\[
A =
\begin{pmatrix}
-1 &  1 &  1 & -1 &  1 &  1 \\
 1 &  1 & -1 &  1 &  1 & -1 \\
 1 & -1 &  1 &  1 &  1 &  1 \\
 1 & -1 & -1 &  1 &  1 &  1 \\
-1 & -1 &  1 &  1 &  1 &  1 \\
 1 &  1 &  1 &  1 & -1 & -1 \\
 1 &  1 &  1 & -1 & -1 &  1
\end{pmatrix}
\]



\end{document}